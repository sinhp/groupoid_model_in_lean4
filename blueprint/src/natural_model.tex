In this section we describe the categorical semantics of
HoTT via Natural Models.
This will not be a detailed account of the syntax of HoTT,
but will be a detailed account of what is needed to interpret such syntax.

\medskip

\begin{notation*}
  We will have two universe sizes - one small and one large.
  We denote the category of small sets as $\set$ and the large sets as $\Set$.
  For example, we could take the small sets $\set$ to be those in $\Set$ bounded in cardinality
  by some inaccessible cardinal.
\end{notation*}

\subsection{Types}

% Assume an inaccessible cardinal $\lambda$. Write $\Set$ for the category of all sets. Say that a set $A$ is $\lambda$-small if $|A| < \lambda$.  Write $\Set_\lambda$ for the full
% subcategory of $\Set$ spanned by  $\lambda$-small sets.

Let $\catC$ be a small category, i.e.~a category whose class of objects is a $\set$ and
whose hom-classes are from $\set$.
% We do not assume that $\mathbb{C}$ is $\lambda$-small for
% the moment.
We write $\pshC$ for the category of presheaves over $\catC$,
\[
\pshC \defeq [\catC^\op, \Set]
\]

\medskip

% Because of the assumption of the existence of $\lambda$, $\pshC$ has additional structure. Let
% \[
% \Term \to \Type
% \]
% be the Hofmann-Streicher universe in $\pshC$ associated to $\lambda$. In particular,
% \[
% \Type(c) \defeq \{  A \co (\catC_{/c})^\op \to \Set_{\lambda} \}
% \]

\begin{defn}
  Following Awodey \cite{awodey2017naturalmodelshomotopytype},
  we say that a map $\tp : \Term \to \Type$ is presentable when
  any fiber of a representable is representable.
  In other words, given any $\Ga \in \catC$ and a map $A : \yo (\Ga) \to \Type$,
  there is some representable $\Ga \cdot A \in \catC$ and maps $\disp{A} : \Ga \cdot A \to \Ga$
  and $\var_{A} : \yo (\Ga \cdot A) \to \Term$ forming a pullback
  % https://q.uiver.app/#q=WzAsNCxbMCwwLCJcXHlvIChcXEdhIFxcY2RvdCBBKSJdLFswLDEsIlxceW8gKFxcR2EpIl0sWzEsMSwiXFxUeXBlIl0sWzEsMCwiXFxUZXJtIl0sWzAsMSwiXFx5byAoXFxkaXNwe0F9KSIsMl0sWzEsMiwiQSIsMl0sWzMsMl0sWzAsMywiXFx2YXJfQSJdXQ==
\[\begin{tikzcd}
	{\yo (\Ga \cdot A)} & \Term \\
	{\yo (\Ga)} & \Type
	\arrow["{\var_A}", from=1-1, to=1-2]
	\arrow["{\yo (\disp{A})}"', from=1-1, to=2-1]
	\arrow[from=1-2, to=2-2]
	\arrow["A"', from=2-1, to=2-2]
\end{tikzcd}\]
\end{defn}

\medskip

The Natural Model associated to a presentable map $\tp \co \Term \to \Type$ consists of
\begin{itemize}
\item contexts as objects $\Gamma, \Delta, \ldots \in \catC$,
\item a type in context $\yo (\Gamma)$ as a map $A \co \yo(\Gamma) \to \Type$,
\item a term of type $A$ in context $\Gamma$ as a map $a \co \yo(\Gamma) \to \Term$ such that
\[
\xymatrix{
 & \Term \ar[d]^{\tp} \\
\Gamma \ar[r]_-A \ar[ur]^{a} & \Type}
\]
commutes,
\item an operation called ``context extension'' which given a context $\Gamma$ and a type $A \co \yo(\Gamma) \to \Type$ produces a context $\Gamma\cdot A$ which fits into a pullback diagram below.
\[
\xymatrix{
\yo(\Gamma.A) \ar[r] \ar[d] & \Term \ar[d] \\
\yo(\Gamma) \ar[r]_-{A} & \Type}
\]
\end{itemize}

% In the internal type theory of $\pshC$, we write
% \[
%  (\Gamma) \ A \co \Type \qquad  (\Gamma) \ a \co A
% \]
% to mean that $A$ is a type in context $\Gamma$ and that $a$ is an element of type $A$ in context
% $\Gamma$, respectively.




{\bf Remark.}
Sometimes, we first construct a presheaf $X$ over $\Gamma$ and observe that it can be classified by a map into $\Type$. We write
\[
\xymatrix@C=1cm{
X \ar[r] \ar[d]& \Term \ar[d] \\
\yo(\Gamma) \ar[r]_-{\ulcorner X \urcorner} & \Type}
\]
to express this situation, i.e.~$X \cong \yo(\Gamma \cdot \ulcorner X \urcorner)$.

\medskip



\subsection{A type of small types}

We now wish to formulate a condition that allows us to have a type of small types, written $\U$, not just {\em judgement} expressing that something is a type. With this notation, the judgements that we would like to derive is
\[
 \U \co \Type \qquad
 \begin{prooftree}
 a \co \U
 \justifies
 \El(a) \co \Type
 \end{prooftree}
\]

(A sufficient and natural condition for this seems to be that we now have another inaccessible cardinal $\kappa$, with $\kappa < \lambda$.)

In the Natural Model, a universe $\U$ is postulated by a map
\[
\pi \co \E \to \U
\]

In the Natural Model:
\begin{itemize}
\item There is a pullback diagram of the form
\[
\xymatrix{
\U \ar[r] \ar[d] & \Term \ar[d] \\
1 \ar[r]_-{\ulcorner \U \urcorner } & \Type }
\]
\item There is an inclusion of $\U$ into $\Type$
\[
\El \co \U \rightarrowtail \Type
\]
\item $\pi : \E \to \U$ is obtained as pullback of $\tp$; There is a pullback diagram
\[
\xymatrix{
E \ar@{>->}[r] \ar[d] & \Term \ar[d] \\
\U \ar@{>->}[r]_-{\El} & \Type }
\]
 \end{itemize}

With the notation above, we get
\[
\xymatrix{
\yo (\Gamma.\El(a)) \ar[r] \ar[d] & \E \ar[r] \ar[d] & \Term \ar[d] \\
\yo (\Gamma) \ar[r]_a  \ar@/_2pc/[rr]_-{A} & \U \ar[r]_{\El} & \Type}
\]
Both squares above are pullback squares.


\subsection{The Universe in Embedded Type Theory (HoTT0) and the relationship to the Natural Model}
