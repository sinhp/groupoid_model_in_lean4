\documentclass[a4paper,11pt]{article}

\usepackage{amsthm,amsmath,amssymb,stmaryrd,enumerate,a4,array}
\usepackage{geometry}
\usepackage{url}
\usepackage{cleveref}
\usepackage[cmtip,all]{xy} 
\usepackage{hyperref}
\numberwithin{equation}{section} 
\usepackage[parfill]{parskip} 

\input{prooftree}

%%%%%%% THEOREMS %%%%%%%%%
\newtheorem{theorem}{Theorem}[section]
\newtheorem*{theorem*}{Theorem}
\newtheorem{prop}[theorem]{Proposition}
\newtheorem{obs}[theorem]{Observation}
\newtheorem{lemma}[theorem]{Lemma}
\newtheorem{cor}[theorem]{Corollary}
\newtheorem{applemma}{Lemma}[section]

\theoremstyle{definition} 
\newtheorem{defn}[theorem]{Definition}

\theoremstyle{remark}
\newtheorem{rmk}[theorem]{Remark}
\newtheorem*{eg}{Example}
\newtheorem{ex}{Exercise}
\newtheorem*{remark*}{Remark}
\newtheorem*{remarks*}{Remarks}
\newtheorem*{notation*}{Notation}
\newtheorem*{convention*}{Convention}

\newenvironment{comment}{\begin{quote} \em Comment. }{\end{quote}}


\newcommand{\defeq}{=_{\mathrm{def}}}
\newcommand{\iso}{\cong}
\newcommand{\equi}{\simeq} 
\newcommand{\retequi}{\unlhd}
\newcommand{\op}{\mathrm{op}}
\newcommand{\Id}{\mathrm{Id}}
\newcommand{\Nat}{\mathbb{N}}
\newcommand{\co}{\colon}


\newcommand{\cat}{\mathbb} 
\newcommand{\catC}{\mathbb{C}}
\newcommand{\tcat}{\mathbf}
\newcommand{\Set}{\tcat{Set}}
\newcommand{\psh}[1]{\mathrm{Psh}(#1)}
\newcommand{\yo}{\mathsf{y}}
\newcommand{\pshC}{\psh{\catC}}
\newcommand{\Type}{\mathsf{Ty}}
\newcommand{\Term}{\mathsf{Tm}}
\newcommand{\tp}{\mathsf{tp}}
\newcommand{\Prop}{\mathsf{Prop}}
\newcommand{\U}{\mathsf{U}}
\newcommand{\E}{\mathsf{E}}
\newcommand{\El}{\mathsf{El}}


\title{Universe in the Natural Model of Type Theory}
\author{Sina Hazratpour}

\date{\today}

\begin{document}


\maketitle



\section{Types}

Assume an inaccessible cardinal $\lambda$. Write $\Set$ for the category of all sets. Say that a set $A$ is $\lambda$-small if $|A| < \lambda$.  Write $\Set_\lambda$ for the full
subcategory of $\Set$ spanned by  $\lambda$-small sets.

Let $\mathbb{C}$ be a small category, i.e.~a category whose class of objects is a set and
whose hom-classes are sets. 
% We do not assume that $\mathbb{C}$ is $\lambda$-small for
% the moment.

We write $\pshC$ for the category of presheaves over $\catC$,
\[
\pshC \defeq [\catC^\op, \Set]
\]

% Because of the assumption of the existence of $\lambda$, $\pshC$ has additional structure. Let 
% \[
% \Term \to \Type
% \]
% be the Hofmann-Streicher universe in $\pshC$ associated to $\lambda$. In particular,
% \[
% \Type(c) \defeq \{  A \co (\catC_{/c})^\op \to \Set_{\lambda} \}
% \]

The Natural Model associated to a presentable map $\tp \co \Term \to \Type$ consists of 
\begin{itemize}
\item contexts as objects $\Gamma, \Delta, \ldots \in \catC$,
\item a type in context $\yo (\Gamma)$ as a map $A \co \yo(\Gamma) \to \Type$,
\item a term of type $A$ in context $\Gamma$ as a map $a \co \yo(\Gamma) \to \Term$ such that
\[
\xymatrix{
 & \Term \ar[d]^{\tp} \\
\Gamma \ar[r]_-A \ar[ur]^{a} & \Type} 
\]
commutes,
\item an operation called ``context extension'' which given a context $\Gamma$ and a type $A \co \yo(\Gamma) \to \Type$ produces a context $\Gamma\cdot A$ which fits into a pullback diagram below.
\[
\xymatrix{
\yo(\Gamma.A) \ar[r] \ar[d] & \Term \ar[d] \\
\yo(\Gamma) \ar[r]_-{A} & \Type}
\]
\end{itemize}

% In the internal type theory of $\pshC$, we write
% \[
%  (\Gamma) \ A \co \Type \qquad  (\Gamma) \ a \co A
% \]
% to mean that $A$ is a type in context $\Gamma$ and that $a$ is an element of type $A$ in context 
% $\Gamma$, respectively.




{\bf Remark.} 
Sometimes, we first construct a presheaf $X$ over $\Gamma$ and observe that it can be classified by a map into $\Type$. We write
\[
\xymatrix@C=1cm{
X \ar[r] \ar[d]& \Term \ar[d] \\
\yo(\Gamma) \ar[r]_-{\ulcorner X \urcorner} & \Type}
\]
to express this situation, i.e.~$X \cong \yo(\Gamma \cdot \ulcorner X \urcorner)$. 

\medskip



\section{A type of small types}

We now wish to formulate a condition that allows us to have a type of small types, written $\U$, not just {\em judgement} expressing that something is a type. With this notation, the judgements that we would like to derive is
\[
 \U \co \Type \qquad 
 \begin{prooftree}
 a \co \U 
 \justifies
 \El(a) \co \Type
 \end{prooftree}
\]

(A sufficient and natural condition for this seems to be that we now have another inaccessible cardinal $\kappa$, with $\kappa < \lambda$.) 

In the Natural Model, a universe $\U$ is postulated by a map 
\[
\pi \co \E \to \U
\]

In the Natural Model: 
\begin{itemize}
\item There is a pullback diagram of the form
\[
\xymatrix{
\U \ar[r] \ar[d] & \Term \ar[d] \\
1 \ar[r]_-{\ulcorner \U \urcorner } & \Type }
\]
\item There is an inclusion of $\U$ into $\Type$ 
\[
\El \co \U \rightarrowtail \Type
\]
\item $\pi : \E \to \U$ is obtained as pullback of $\tp$; There is a pullback diagram
\[
\xymatrix{
E \ar@{>->}[r] \ar[d] & \Term \ar[d] \\
\U \ar@{>->}[r]_-{\El} & \Type }
\]
 \end{itemize}

With the notation above, we get
\[
\xymatrix{
\yo (\Gamma.\El(a)) \ar[r] \ar[d] & \E \ar[r] \ar[d] & \Term \ar[d] \\
\yo (\Gamma) \ar[r]_a  \ar@/_2pc/[rr]_-{A} & \U \ar[r]_{\El} & \Type}
\]
Both squares above are pullback squares.


\section{The Universe in Embedded Type Theory (HoTT0) and the relationship to the Natural Model}

\end{document}

